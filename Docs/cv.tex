\documentclass[11pt,a4paper]{moderncv}
\moderncvtheme[orange]{classic}
\usepackage[utf8]{inputenc}
\usepackage[top=1.0cm, bottom=1.1cm, left=1.5cm, right=1.5cm]{geometry}
\newcommand{\ts}{\textsuperscript}
%\usepackage{matt}
% Largeur de la colonne pour les dates
\setlength{\hintscolumnwidth}{3.0cm}


\firstname{Matthieu}
\familyname{Riegler}
\title{Ingénieur de développpement logiciel}
\photo[96pt][0.4pt]{photo2.jpg}
\address{44 Rue Bizanet}{38000 Grenoble}
\email{matthieu@riegler.fr}
\homepage{www.riegler.fr}
\mobile{06 16 50 07 20}
\extrainfo{23 ans -- Permis B}
%\social[twitter]{Kyro38}
%\quote{Expect the best, plan for the worst, and prepare to be surprised}
\begin{document}
\maketitle

\section{Formation}
\cventry{2011 -- 2013}{Master Mathématiques et Informatique}{Université Joseph Fourier, Grenoble (38)}{}{}{Filière Génie Informatique — spécialisation en IHM et technologies mobiles}
\cventry{2007 -- 2011}{Licence d'informatique}{Université Joseph Fourier, Grenoble (38)}{}{}{}
\cventry{2004 -- 2007}{Baccalauréat Scientifique}{Lycée International, Grenoble (38)}{}{}{Section allemande, obtention du diplome d'allemand KMK (équivalent C1)}

\section{Expériences professionelles}
\cventry{Avril 2013\\à Sept 2013\\(6 mois)} {Stage de fin d'études} {Orange Business Services / IT\&Labs, Montbonnot (38)} {}{} {
\begin{itemize}
\item {Stage R\&D pour le compte d'\emph{Orange Labs} dans le FabLab \emph{Thinging!}}
\item {Reflexion menée sur l'Internet des objets et définition des projets à mener }
\item {Réalisation d'une application mobile Android (Java)}
\item {Réalisation d'une application serveur en Python avec le framework Django}
\item {Gestion de projet en agilité/Méthode Scrum. Réalisation de la documentation de génie logiciel.}
\end{itemize}}
\cventry{Fév 2012\\à Juin 2012}{Stagiaire}{Laboratoire d'Informatique de Grenoble, Équipe IIHM}{Grenoble (38)}{}{
\begin{itemize}
\item{TER, sujet : « Techniques de sélection en réalité augmentée sur dispositif mobile » }
\item{Développement d’une application de réalité augmentée sur iPhone/iPad (Objective-C, OpenGL).}
\item{Réalité augmentée basée sur le Framework Qualcomm Vuforia (QCAR)}
\end{itemize}}
\cventry{Sept 2011\\à Juin 2013}{Tuteur référent} {Université Joseph Fourier}{Grenoble (38)}{} {Chargé de tutorat pour des 2\ts{e} année de Licence, d'écriture et de correction des examens de C2i.}
\cventry{Juil 2008\\(1 mois)}{Ouvrier}{Südzucker}{Ochsenfurt (Allemagne)}{}{Ouvrier chargé d'approvisionnement en sucre.}

\section{Compétences}
\cvitem{Systèmes} {OSX, GNU/Linux (Ubuntu), Windows (XP, 7), iOS, Android.}
\cvitem{Langages} {C/C++/Objective-C, Java/JEE, Python, scripting Bash, SQL, LaTeX, XML, CSS.}
\cvitem{Conception} {UML, Design pattern, conception orientée objet, gestion de versions.}
\cvitem{Logiciels} {Eclipse, Netbeans, Emacs, XCode, MS Office, Photoshop/Lightroom.}
\cvitem{Frameworks} {Cocoa, Qt, Swing, OpenGL, Django.}
\cvitem{IHM} {Multimodalité, Ergonomie des interfaces, Modèles de tâches, Vision par ordinateur et traitements d'images.}
\cvitem{Autre} {Gestion de projet, travail en équipe, méthodes agiles, techniques de tests, gestion de versions (svn, git), veille informatique.}
\cvitem{Langues} {\textbf{Anglais} : Bon niveau. Lu, écrit, parlé.\newline{}%
\textbf{Allemand} : Bilingue – lu, écrit et parlé couramment – Section allemande en Lycée International – séjours réguliers en Allemagne depuis l'enfance.}


\section{Centres d'intérêt}
\begin{itemize}
\item {Photographie (portrait), cinéma en version originale (thriller, policier, comédie), bricolage.}
\item {Cyclisme sur route, cyclotourisme (Francfort-Grenoble).}
\item {Contributeur \& administrateur de la Wikipédia francophone depuis 2007. Organisation des ateliers Wikipédia de Grenoble en partenariat avec le CCSTI.}
\item {}
\end{itemize}
\end{document}
